\section{信号処理方法による統計誤差の改善}
火山の熱水系構造や活動状況, 火口からの距離などにより, 
\ce{SO2}濃度は変動する. 
三宅島における観測値は, 2カ月間(平成13年2, 3月)の平均濃度が2.11 ppmであるのに対して, 
最高値は11.41 ppmが報告されている. 
また, 霧島火山における観測値は, 新燃岳が25.6 ppm, 硫黄山で228 ppmが報告されている. 
提案手法であるラマンDIALにおいて, 
SO2が高濃度, または広範囲に分布する環境で測定すると, 
信号の減衰が大きくなり, 
統計誤差も悪化する.
そのため, \ce{SO2}の濃度変動に伴う統計誤差の悪化を考慮したシステムの設計が必要となる. 

% 前章では, 
% 火口直上の濃度分布を正確にとらえたい場合は
% さらに細かい距離分解能が求められることがわかった. 
% しかし, 統計誤差は\cref{equ:stat_err}より, 
% 距離分解能が大きいほど抑制される相反関係にある. 
DIAL観測では, ライダー信号は細かい距離分解能取得し, 
濃度推定の際は信号強度に合わせて距離分解能の調整が行われる.
距離分解能$\Delta R$は\cref{equ:power}におけるシステムパラメータの中の比例係数であり, 
距離分解能の調整により, 信号加算を行うことで$P$は増大し, 
\cref{equ:stat_err}に示された統計誤差は調整前より減少する. 
本章では, \ce{O2}ストークス散乱と\ce{O2}アンチストークス散乱によるラマンDIALについて, 
距離分解能調整の有無による観測を比較評価した. 

特に火山ガス放出が多い日の観測値を参考に, 
濃度分布は前章で使用したものについて,   
高濃度区間の\ce{SO2}を30 ppm から 100 ppm に変更し, 
距離分解能が5 m と25 mの2つでシミュレーションを行った. 
\cref{fig:res_staterr}にそれぞれのシミュレーション結果を示す. 

\begin{figure}
    \centering
    \includegraphics[width=1\linewidth]{images/sim_02_staterr.pdf}
    \caption{$\Delta R=5, 25$の場合の二酸化硫黄測定シミュレーション結果}
    \label{fig:res_staterr}
\end{figure}

\noindent
\cref{fig:res_staterr}より, どちらの測定でも, 
視線距離が300 m 及び700 m周辺にある濃度の切り替わりで誤差が発生するが, 
$\Delta R$が25 mの測定は追従性の不足がより顕著である. 
一方で, 統計誤差は$\Delta R$が5 mの場合, 697.5m地点で13.4 ppmであるのに対して, 
$\Delta R$が25 mの場合, 697.5m地点で1.16 ppmに改善された.
これは, 設定値の100 ppmに対して10 \%以下を十分に満たしているため, 
距離分解能の調整で実測定に耐えるシステムとなることが示された. 
そのため, 実観測では高距離分解能でライダー信号を取得した上で, 
濃度推定時に信号強度や濃度勾配に応じて
距離分解能を動的に変更する処理が有効であると考えられる.
