\begin{thebibliography}{9}\footnotesize
% \bibitem{tarou1} 都立太郎, ``特別研究'' 東京都立大学システムデザイン学部., VOL.1, NO.1, pp.1-2, 2008年12月.

\bibitem{Previous_research} 伊藤, "特別研究", 東京都立大学システムデザイン学部, 2022
\bibitem{PowerEqu} Thomas J. McGee,  "Raman DIAL measurements of stratospheric ozone in the presence of volcanic aerosols", (American Geophysical Union 1993), p. 956

\bibitem{DIALEqu} S.Ismaiil, "Airborne and spaceborne lidar measurements of water vapor profiles: a sensitivity analysis", (APPLIED OPTICS vol. 28 no.17 1989), p. 3604.

\bibitem{miyake} 内閣府, "三宅島火山ガスに関する検討会報告書", (2003)

\bibitem{Kirishima}
大場武, 野上健治, 平林順一, "霧島火山地帯の温泉水と噴気ガスの化学組成と同位体比から推定される熱水系", 火山, 42, (1997), 1-15

\bibitem{ramanXS} 清水浩, 小林喬郎, 稲葉文男, "気体分子のラマン散乱断面積の測定", 応用物理, 42, (1973), 889-898

\bibitem{pasquill} Frank Pasquill, "Atmospheric Diffusion: The Dispersion of Windborne Material from Industrial and Other Sources", D. Van Nostrand Company, London, 1962.

\bibitem{gifford} Frank A. Gifford Jr., “Use of routine meteorological observations for estimating atmospheric dispersion”, Nuclear Safety, Vol. 2, No. 4, 1961, pp. 47–51.

\end{thebibliography}
