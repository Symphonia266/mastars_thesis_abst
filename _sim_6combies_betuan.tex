\section{最適なラマン散乱波長の検討}\label{sec:6combi}
ラマンDIALによる測定で発生する統計誤差は\cref{equ:stat_err}で求められる. 
なお, 背景光による影響は無視している. 
% 式中の$D$はダークカウント, $B_j$は背景光雑音, $F$はディテクターの雑音指数を表す. 
\begin{align}
    \varepsilon_{\mathrm{stat}} 
    &= \dfrac{1}{2\Delta{\sigma}{SO2}\Delta R}\left[
        \sum^2_{i=1}\sum^2_{j=1}
        \frac{1}{P_{ij}}
    \right]
    \label{equ:stat_err}
\end{align}
i=1,2は視線距離$R_1, R_2$であり, 
j=1,2はon波長及びoff波長を表す
\cref{equ:stat_err}より, 
\staterr は$P$と$\Delta\subrm{\sigma}{SO2}$に依存することがわかる.
吸収が過度に強い波長では信号が減衰し$P$が小さくなる一方,
吸収が弱い波長では$\Delta\subrm{\sigma}{SO2}$が小さくなるため,
いずれの場合も統計誤差は増大する.

二酸化硫黄の吸収スペクトルを\cref{fig:absorptance_spectrum}に示す. 
縦軸は消散係数($=\subrm{n}{gas}(R)\times\subrm{\sigma}{gas}(\lambda)$)である. 
二酸化硫黄は紫外領域に吸収スペクトルを持ち, 
300 nm付近の波長では吸収が櫛状に変動する特徴を持つ.
二酸化硫黄測定にラマンDIALを適応する場合, 
\cref{equ:stat_err}の$P$と$\Delta{\sigma}{SO2}$はon波長とoff波長に依存し, 
on波長及びoff波長に相当するラマン散乱波長はレーザ波長に依存する. 
したがって, 信号強度と差分吸収が十分に確保できるレーザ波長が存在するといえる. 
\begin{figure}
    \centering
    \includegraphics[width=\linewidth]{images/sim_01_3gases_absorptance}
    \caption{二酸化硫黄,硫化水素,オゾンの吸収スペクトル}
    \label{fig:absorptance_spectrum}
\end{figure}

酸素分子および窒素分子のラマン散乱波長には,
それぞれストークス散乱波長とアンチストークス散乱波長の2種類が存在し, 
レーザ波長に対して生じる4つのラマン散乱波長から
2波長を選択する場合,
組み合わせは$\,_4\mathrm{C}_2=6$通りとなる.
先行研究ではNd:YAGの第4高調波である266 nmに対する
酸素ストークス散乱波長と窒素ストークス散乱波長による
ラマンDIALについて検討した\cite{Previous_research}. 
しかし, \cref{fig:absorptance_spectrum}に示す通り, 
\ce{SO2}の吸収は240 nm付近や320 nm以降で吸収が弱まるため, 
レーザ波長とラマン散乱波長の関係によっては
アンチストークス散乱波長の利用が優位となる可能性がある. 

以上を踏まえ,
本研究では6通りすべてのラマン散乱波長の組み合わせについて,
二酸化硫黄測定における統計誤差のシミュレーションを行い,
最適なレーザ波長およびラマン散乱波長の組み合わせを検討した.
観測距離は1 kmまで,
気体の濃度分布は三宅島での平均的な火山ガス濃度観測値を参考に, 
二酸化硫黄が高濃度区間で30 ppm, 低濃度区間で0.07 ppm, 
オゾンが観測領域全体で0.005 ppm, 
硫化水素は二酸化硫黄の 1/2 とした.
各気体の吸収断面積の温度・圧力依存性を考慮し, 
ライダー設置高度を1 kmとした.
気体濃度は三宅島における平均的観測値を基に設定した\cite{miyake}.
その他のシステム条件は,
レーザ出力10 mJ,
受信鏡直径0.3 m,
パルス周波数100 Hzで10分間積算,
距離分解能5 mとした.

6通りのラマン散乱波長の組み合わせに対する
二酸化硫黄測定のシミュレーション結果を\cref{fig:res_6combi}に示す.
\begin{figure}[htbp]
    \centering
    \includegraphics[width=\linewidth]{images/sim_01_n_SO2_6combies_stat_zoom}
    \caption{ラマン散乱波長の組み合わせ毎の二酸化硫黄測定シミュレーション結果}
    \label{fig:res_6combi}
\end{figure}
% \begin{table}[htbp]
%     \centering
%     \begin{tabular}{c|c}
%         Combination  & Laser wavelength [nm]\\\hline\hline
%         O2 Stokes, O2 a-Stokes & 334.6 \\
%         N2 Stokes, O2 a-Stokes & 334.6 \\
%         O2 Stokes, N2 a-Stokes & 343.7 \\
%         N2 Stokes, N2 a-Stokes & 338.9 \\
%         N2 a-Stokes, O2 a-Stokes & 343.6 \\
%         N2 Stokes, O2 Stokes & 240.0\\
%     \end{tabular}
%     \caption{Caption}
%     \label{tab:6combi_ls_wl}
% \end{table}

\noindent
その結果,
観測距離700 m地点における統計誤差が最小となる条件は,
レーザ波長334.6 nm,
酸素アンチストークス散乱波長(318.0 nm)をon波長,
酸素ストークス散乱波長(353.0 nm)をoff波長とする波長構成であることがわかった.

\setcounter{inlineitem}{0}
先行研究で提案した
 \inlineitem{itm:st-st} 酸素ストークス散乱波長と窒素ストークス散乱波長の構成と, 
本研究で最適と判断された 
\inlineitem{itm:st-ast} 酸素ストークス散乱波長と酸素アンチストークス散乱波長の構成について, 
各波長における
受信強度の距離特性を\cref{fig:res_cond_power}に示す. 
\begin{figure}[htbp]
    \centering
    \includegraphics[width=\linewidth]{images/sim_01_power.pdf}
    \caption{2構成の各波長における受信強度の距離特性}
    \label{fig:res_cond_power}
\end{figure}

\noindent
\cref{itm:st-st}では,
on波長, off波長ともに吸収が強いため, 高濃度区間での減衰が著しいが, 
\cref{itm:st-ast}では,
不要な吸収を抑え, 遠方での信号強度は十分確保できている. 
また, $\Delta\subrm{\sigma}{SO2}$についても, 
\cref{itm:st-st}は$6.83\times10^{-24} \mathrm{m}^2$であるのに対して, 
\cref{itm:st-ast}は$7.22\times10^{-24} \mathrm{m}^2$と僅かに大きい. 

以上より, 
本研究で設定した条件下では, 
レーザ波長に334.6 nmを使用し, 
酸素アンチストークス散乱波長をon波長,
酸素ストークス散乱波長をoff波長とする
ラマンDIALが統計誤差的に最も優れることが示された. 
これは, 当該波長構成が差分吸収を十分に確保しつつ, 
共存気体による不要な吸収を抑制できるためであり, 
吸収スペクトル形状および共存気体の影響を考慮した上での
アンチストークス散乱波長の利用が有効であることが明らかになった.
