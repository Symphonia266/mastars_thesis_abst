\documentclass[a4paper]{ltjsarticle}
\usepackage{fontspec}
\usepackage{luatexja}
\usepackage{luatexja-fontspec}
% Set font - try Harano Aji first, then Noto Serif JP, then fall back to Yu Mincho
% \setmainjfont{Noto Sans JP} % 環境に合わせてフォントを調整する必要あり

% 欧文(Computer Modern 互換)
\setmainfont{Latin Modern Roman}
\setsansfont{Latin Modern Sans}
\setmonofont{Latin Modern Mono}

% 和文(jsarticle 既定相当)
\setmainjfont[
    UprightFont   = {Noto Sans JP Regular},   % 通常
    BoldFont      = {Noto Sans JP Bold},      % 太字
    MediumFont    = {Noto Sans JP Medium},    % 中太
    SemiBoldFont  = {Noto Sans JP SemiBold},  % セミボールド
    BlackFont     = {Noto Sans JP Black},     % 黒字
    LightFont     = {Noto Sans JP Light},     % 細字
    ExtraLightFont= {Noto Sans JP ExtraLight},% 極細
    ExtraBoldFont = {Noto Sans JP ExtraBold}  % 超太字
]{Noto Sans JP Regular}

\usepackage{eecspresen_m2_mo}
\usepackage{graphicx}
\usepackage{siunitx}
\usepackage{float}
\usepackage{amsmath}
\usepackage{hyperref}
\usepackage{cleveref}
\usepackage{enumitem}
\usepackage{caption}

\usepackage{subcaption}
\usepackage[version=4]{mhchem}
\usepackage{setspace}
\usepackage{titlesec}
\usepackage{indentfirst}
% \usepackage[backend=biber,style=numeric,sorting=nyt]{biblatex}
% \usepackage{cite}

\usepackage[british, english]{babel}
% babelパッケージにjapaneseは無いらしい
% 以下のDefineBiblio~コマンドを使いたいけれど言語設定が必要なので、
% 便宜的にjapaneseの代わりにbritishを割り当てておく
% \DefineBibliographyStrings{british}{andothers={他}} % 和文の文献なら「他」
% \DefineBibliographyStrings{english}{andothers={\textit{et al.}}} % 斜体にする

% \AtEveryBibitem{ % bibファイルの文献を走査してゆく
%   \iffieldequalstr{langid}{Japanese}{ % if: langid=Japaneseとした文献のみ取得
%     \selectlanguage{british} % 言語をbritishに設定
%     \DeclareDelimFormat{finalnamedelim}{ % 連名の最後に「and」を使わせない
%       \ifnumgreater{\value{liststop}}{2}{\finalandcomma}{}
%     \addspace\multinamedelim }
%     \DeclareFieldFormat{title}{\textrm{「#1」}} % 論文タイトルを鍵括弧
%     \DeclareFieldFormat[book]{title}{\textrm{『#1』}} % 書籍タイトルを二重鍵括弧
%     \DeclareFieldFormat{journaltitle}{\textrm{#1}} % 雑誌タイトルを斜体解除
%     \DeclareFieldFormat{booktitle}{\textrm{#1}} % 書籍タイトルを斜体解除
%     \renewbibmacro{in:}{} % ジャーナル名の前の「In:」を除去
%   }{% ここまでが和文の文献に限った処理
%   \selectlanguage{english}} % else: 欧文の文献はデフォルト(english)の処理で良い
% }

% % use workspace bib file
% \addbibresource{reference.bib}
% % 見出しを section* 相当に
% \defbibheading{bibliography}[\refname]{%
%   \section*{\center\normalsize #1}%
% }

% ラベルを [1] 形式に
% \DeclareFieldFormat{labelnumberwidth}{[#1]\hspace{0.5em}}


% % 左マージン調整 (thebibliography 相当) -- undefined lengths guarded
% \makeatletter
% \@ifundefined{biblabelsep}{}{\setlength{\biblabelsep}{0pt}}
% \@ifundefined{biblabelwidth}{}{\setlength{\biblabelwidth}{0pt}}
% % \@ifundefined{leftmargin}{}{\setlength{\leftmargin}{0pt}}

% % 各文献間の詰め
% \@ifundefined{bibitemsep}{}{\setlength{\bibitemsep}{0pt}}
% \@ifundefined{bibparsep}{}{\setlength{\bibparsep}{0pt}}
% \makeatother


% 行間を詰める
\renewcommand{\baselinestretch}{0.85}
\titlespacing*{\section}
  {0pt}
  {0.5\baselineskip}
  {0.5\baselineskip}

% 図表周りの空間調整
\renewcommand{\textfloatsep}{5pt}  % 図表と本文の間隔(ページ上端・下端)
\renewcommand{\floatsep}{5pt}      % 図表同士の間隔
\renewcommand{\intextsep}{5pt}     % 本文中に置かれた図表との間隔
\renewcommand{\belowcaptionskip}{5pt}
\captionsetup{
  skip=5pt,
  labelfont=bf,
  labelsep=period,
  figurename=図,
  tablename=表
}
% cleveref names and formats (match project defaults)
% --- 図表名を日本語に固定 ---
\renewcommand{\figurename}{図}
\renewcommand{\tablename}{表}

\crefname{figure}{図}{図}
\Crefname{figure}{図}{図}

\crefname{table}{表}{表}
\Crefname{table}{表}{表}

\crefname{equation}{式}{式}
\Crefname{equation}{式}{式}

\crefname{inlineitem}{}{}
\Crefname{inlineitem}{}{}

\crefformat{equation}{(#2#1#3)式}
\Crefformat{equation}{(#2#1#3)式}
\crefrangeformat{equation}{(#3#1#4)–(#5#2#6)式}

\crefmultiformat{equation}
  {(#2#1#3)式}
  {, (#2#1#3)式}
  {, (#2#1#3)式}
  {, (#2#1#3)式}
\crefrangemultiformat{equation}
  {(#3#1#4)–(#5#2#6)式}
  {, (#3#1#4)–(#5#2#6)式}
  {, (#3#1#4)–(#5#2#6)式}
  {, (#3#1#4)–(#5#2#6)式}

\crefformat{inlineitem}{%
  \textcircled{\scriptsize#2#1#3}%
}

% Keep same float settings as original
\floatplacement{figure}{H}
\floatplacement{table}{H}

\newcounter{inlineitem}
\newcommand{\inlineitem}[1]{%
	\refstepcounter{inlineitem}%
	\label{#1}%
	\space\textcircled{\scriptsize\theinlineitem}%
}
\newcommand{\dd}{\mathop{}\!\mathrm{d}}
\newcommand{\subrm}[2]{\ensuremath{#1_{\mathrm{#2}}}}
\newcommand{\cntmerr}{\ensuremath{\varepsilon_{\mathrm{cntm}}}}
\newcommand{\staterr}{\ensuremath{\varepsilon_{\mathrm{stat}}}}

% Input metadata (title/author macros)
\No{-}
\Jtitle{
    火山ガス中二酸化硫黄濃度分布測定のための\\
    ラマンDIALの検討
    }
\Etitle{
    Simulation study 
    on Raman DIAL system\\
    for volcanic sulfur dioxide measurement
}
\Author{
    24861657
    ___
    平山_拓道
    _______________
    指導教官
    __
    柴田_泰邦_教授
}
\presenyear{2025}
\presendate{2026年2月12日,13日}
\master   % 修士論文発表会の場合
%\bachelor % 特別研究発表会の場合


\begin{document}
\mktitle

% Include body sections from the project
\section{はじめに}
日本は世界全体の約7\%を占める活火山が存在する火山大国であり, 
火山ガスによる人的被害を防ぐために監視が行われている. 
その中でも, 二酸化硫黄は火山ガスの中でも特に強い毒性を持ち, 
2 ppmが許容濃度として設定されており, 
三宅火山をはじめとする様々な火山で測定が行われている. 

火山ガス観測には, 
ガスと触媒との電気化学反応量から濃度を測定する電気化学式センサが広く用いられている. 
しかし, この手法は定点観測を前提としているため, 
気体濃度の空間分布を把握することが困難である. 
さらに, 測定器が定常状態に達するまで濃度を評価できないという制約があり, 
時間的に連続した観測が困難であるという課題を有している.
% 火山ガスの有害性から観測者の安全を確保する遠隔観測手法が求められている. 
そこで,  微量気体の遠隔かつ分布観測が可能な手法として, 
差分吸収ライダー(DIAL:Differential Absorption Lidar)が期待されている. 

DIALは測定気体の吸収が異なる2波長レーザを照射し, 
その受信量の差から濃度分布を推定する手法である. 
電気化学式センサと比較して, 遠隔で時間的に連続して, 周辺の濃度分布を測定可能という点で優れている. 
本研究で提案するラマンDIALは, 
% システムに用いる2波長を送信光によるラマン散乱光とし, 
送信レーザの波長数を削減することでシステムの小型化が可能である. 
% 本研究では, 火山ガス中の二酸化硫黄濃度をラマンDIALで観測することを目標とした.
\section{ラマンDIAL}

ラマンDIALは, 気体分子固有のラマン散乱光を用いて, 
測定気体の吸収量の差から濃度分布を推定する手法である. 
この手法の特徴は, 送信レーザの波長を切り替える必要がなく, 
受信波長のみで吸収量の異なる2波長を構成できる点にある. 
ラマンDIALの基本構成を\cref{fig:ramanDIAL}に示す. 
\begin{figure}[H]
    \centering
    \includegraphics[width=0.8\linewidth]{images/ramanDIAL}
    \caption{ラマンDIAL構成図}
    \label{fig:ramanDIAL}
\end{figure}

ラマンDIALは, 
照射したレーザ光に対して大気中の酸素分子と窒素分子から生じる2つのラマン散乱光の吸収量の差分から
気体濃度を測定する遠隔計測システムである. 
ここで, 対象ガスによってレーザ光が強く吸収される波長をon波長, 吸収されにくい波長をoff波長という. 

ラマン散乱光は入射光と異なる波長をもつ散乱光であり, 
この波長差は分子ごとに固有である. 
また, ラマン散乱には長波長側にシフトするストークス散乱と, 
短波長側にシフトするアンチストークス散乱が存在し, 
一般的にアンチストークス散乱の強度はストークス散乱と比較して1/10以下と非常に微弱である. 

ラマン散乱光の受信光子数は\cref{equ:power,equ:alpha}で得られる\cite{PowerEqu}. 
% これにより, 受信光子数は視線上のあらゆる気体吸収によって減衰することを表している. 

% ===詳細バージョン=== 
% \begin{multline}
%     P(\lambda,R)=\frac{E_0}{h\lambda_{\rm{ls}}}M \eta q \Delta R \frac{A}{R^2}O(R) \beta_{\rm{ram}}\\
%     \exp\left[
%         \int^{R}_{0}\alpha(r, \lambda_{\mathrm{ls}})+\alpha(r, \lambda_{\mathrm{ram}})\dd r 
%     \right]\label{equ:power}
% \end{multline}

\begin{align}
    P(\lambda,R)
    &=C\cdot \beta_{\mathrm{ram}}
    \exp\left[
        \int^{R}_{0}\alpha(r, \lambda_{\mathrm{ls}})+\alpha(r, \lambda_{\mathrm{ram}})\dd r 
    \right]\label{equ:power}\\    
    \alpha(\lambda, r) 
    &= \alpha_{\mathrm{air}}(\lambda)
    +\sum^{N}_{\mathrm{gas}=1}
        n_{\mathrm{gas}}(r)\sigma_{\mathrm{gas}}(\lambda)\label{equ:alpha}
\end{align}
% 式中の$E_0$はパルスエネルギー, 
% $h$はプランク定数, 
% $M$は積算回数, 
% $q$は光学系の全効率, 
% $\eta$はディテクターの量子効率, 
% $\Delta R$は距離分解能, 
% $A$は受信鏡直径
% $O(R)$は重なり関数を表す. 
% 後方ラマン散乱断面積$\beta_{ram}$について, 
% 窒素による$\beta_{ram}$は$2.8\times10^{-34} \mathrm{m}^2$, 
% 酸素による$\beta_{ram}$は$3.9\times10^{-34} \mathrm{m}^2$とし\cite{ramanXS},
% これを0.1した値をアンチストークス散乱の後方散乱断面積とした. 
式中の$C$はライダーのシステムパラメータ, 
$\beta_{ram}$は後方ラマン散乱断面積,  
$\lambda_{\mathrm{ls}}$はレーザ波長, 
$\lambda_{\mathrm{ram}}$はラマン散乱波長である.  
\cref{equ:alpha}に示す消散係数$\alpha$は, 大気及び各種気体による吸収の総和であり, 
ここで$\mathrm{gas}=1,2,…,N$ はガス成分の種類を表す. 

ライダーの受信光子数から二酸化硫黄の濃度分布は\cref{equ:n_SO2}で求められる\cite{DIALEqu}.
$\Delta\sigma$はon波長とoff波長における
二酸化硫黄の吸収断面積の差分である. 

\begin{equation}
    n_{\mathrm{SO2}}
    =\frac{1}{\Delta\sigma_{\mathrm{SO2}}\Delta R}\ln\left(
        \frac
        {P(\lambda_{\mathrm{on}}, R_1)P(\lambda_{\mathrm{off}}, R_2)}
        {P(\lambda_{\mathrm{on}}, R_2)P(\lambda_{\mathrm{off}}, R_1)}
    \right)\label{equ:n_SO2}
    % \Delta\sigma_{\mathrm{SO2}} &= 
    % \sigma_{\mathrm{SO2}}(\lambda_{\mathrm{on}})-
    % \sigma_{\mathrm{SO2}}(\lambda_{\mathrm{off}})
\end{equation}

% \begin{table}[t]
%   \centering
%   \caption{System parameters for Raman DIAL.}
%   \label{tab:system_parameters}
%   \begin{tabular}{c p{60mm} p{20mm}}
%     \hline
%     $E_0$ [mJ] & Pulse energy & 30 \\
%     $A$ [m] & Diameter of the telescope & 0.6 \\
%     $M$ & Number of integration times & $100 \si{Hz}\times$30 \si{min} \\
%     $\eta$ & Quantum efficiency of the detector & 0.3 \\
%     $q$ & Total efficiency of optics & 0.3 \\
%     $\Delta R$ [m] & Spatial resolution & 30 \\

%     % $\sigma_{\mathrm{RamanN2}}$ [m$^2$] &
%     % Stokes Raman scattering cross section of N$_2$ &
%     % $2.8 \times 10^{-34}$ \\

%     % $\sigma_{\mathrm{RamanO2}}$ [m$^2$] &
%     % Stokes Raman scattering cross section of O$_2$ &
%     % $3.9 \times 10^{-34}$ \\

%     % $\sigma'_{\mathrm{ramN2}}$ [m$^2$] &
%     % anti-Stokes Raman scattering cross section of N$_2$ &
%     % $2.8 \times 10^{-35}$ \\

%     % $\sigma'_{\mathrm{ramO2}}$ [m$^2$] &
%     % anti-Stokes Raman scattering cross section of O$_2$ &
%     % $3.9 \times 10^{-35}$ \\
%     \hline
%   \end{tabular}
% \end{table}

\section{最適なラマン散乱波長の検討}\label{sec:6combi}
ラマンDIALによる測定で発生する統計誤差は\cref{equ:stat_err}で求められる. 
なお, 背景光による影響は無視している. 
% 式中の$D$はダークカウント, $B_j$は背景光雑音, $F$はディテクターの雑音指数を表す. 
\begin{align}
    \varepsilon_{\mathrm{stat}} 
    &= \dfrac{1}{2\Delta{\sigma}{SO2}\Delta R}\left[
        \sum^2_{i=1}\sum^2_{j=1}
        \frac{1}{P_{ij}}
    \right]
    \label{equ:stat_err}
\end{align}
i=1,2は視線距離$R_1, R_2$であり, 
j=1,2はon波長及びoff波長を表す
\cref{equ:stat_err}より, 
統計誤差\staterr は$P$と$\Delta{\sigma}{SO2}$に依存することがわかる.   
\ce{SO2}吸収が強すぎる波長では$P$が減衰する一方で, 
吸収が弱すぎる波長では$\Delta{\sigma}{SO2}$が小さくなるため, 
それぞれ\staterr は増大する. 

二酸化硫黄の吸収スペクトルを\cref{fig:absorptance_spectrum}に示す. 
縦軸は消散係数($=\subrm{n}{gas}(R)\times\subrm{\sigma}{gas}(\lambda)$)である. 
二酸化硫黄は紫外領域に吸収スペクトルを持ち, 
300 nm付近の波長では吸収が櫛状に増減する特徴を持つ.
二酸化硫黄測定にラマンDIALを適応する場合, 
\cref{equ:stat_err}の$P$と$\Delta{\sigma}{SO2}$はon波長とoff波長に依存し, 
on波長及びoff波長に相当するラマン散乱波長はレーザ波長に依存する. 
したがって, 信号強度と差分吸収が十分に確保できるレーザ波長が存在するといえる. 
\begin{figure}
    \centering
    \includegraphics[width=\linewidth]{images/sim_01_3gases_absorptance}
    \caption{二酸化硫黄,硫化水素,オゾンの吸収スペクトル}
    \label{fig:absorptance_spectrum}
\end{figure}

また, 受信するon波長とoff波長の構成にも注意する必要がある. 
酸素分子によるラマン散乱波長はストークスシフトとアンチストークスシフトの2種類存在し, 
窒素分子についても同様である. 
レーザ波長に対して生じる4つの波長から受信する2波長を構成する場合, 
取り得る組み合わせは6通りとなる. 
先行研究ではNd:YAGの第4高調波である266 nmに対する
酸素ストークス散乱波長と窒素ストークス散乱波長によるラマンDIALについて検討した\cite{Previous_research}. 
しかし, \cref{fig:absorptance_spectrum}に示す通り, 
\ce{SO2}の吸収は240 nm付近や320 nm以降で吸収が弱まるため, 
レーザ波長とラマン散乱波長の関係によってはアンチストークス散乱波長の利用が優位となる可能性がある. 

以上より, 本研究ではこれら6通りのラマンDIALについて統計誤差をシミュレーションし, 
二酸化硫黄測定に最適なレーザ波長とラマン散乱波長を探索した. 

シミュレーションについて, 
観測領域はライダーから距離1 kmまでとし, 
火山ガスの分布は300 mから700 mの間を高濃度区間, その前後を低濃度区間に設定した. 
また, 各気体の吸収断面積$\sigma$の気温, 気圧依存性を考慮し, 
ライダー設置高度を1 kmとした.
紫外領域で光吸収がある気体の濃度分布は三宅島での平均的な火山ガス濃度観測値を参考に, 
二酸化硫黄が高濃度区間で30 ppm, 低濃度区間で0.07 ppm, 
オゾンが観測領域全体で0.005 ppmとし, 
硫化水素は二酸化硫黄の 1/2 とした\cite{miyake}. 
システムパラメータ$C$のうち, 
ライダーのレーザ出力は10 mJ, 
受信鏡の直径は0.3 m, 
積算回数はパルス周波数が100 Hzのレーザを10分間積算したもの, 
距離分解能は5 mとした. 
% 本研究では波長選択の相対比較に主眼を置くため背景光雑音とダークカウント, 
% ディテクターの雑音指数は無視した. 
この条件からライダーの受信光子数を計算し, 統計誤差を導出する. 
二酸化硫黄の吸収スペクトルがある240 nmから370 nmの間で, 
6つのラマンシフトの組み合わせそれぞれで, 
観測領域の最遠方となる995 m地点での統計誤差が最小となるレーザ波長を探索し, 
その中で最適なラマンシフトの組み合わせを決定した. 
% ラマンシフトの組み合わせ毎の統計誤差が最小となったレーザ波長を\cref{tab:6combi_ls_wl}に示す. 
6通りのラマンシフトの組み合わせ毎の二酸化硫黄測定のシミュレーション結果を\cref{fig:res_6combi}に示す. 

% \begin{table}[htbp]
%     \centering
%     \begin{tabular}{c|c}
%         Combination  & Laser wavelength [nm]\\\hline\hline
%         O2 Stokes, O2 a-Stokes & 334.6 \\
%         N2 Stokes, O2 a-Stokes & 334.6 \\
%         O2 Stokes, N2 a-Stokes & 343.7 \\
%         N2 Stokes, N2 a-Stokes & 338.9 \\
%         N2 a-Stokes, O2 a-Stokes & 343.6 \\
%         N2 Stokes, O2 Stokes & 240.0\\
%     \end{tabular}
%     \caption{Caption}
%     \label{tab:6combi_ls_wl}
% \end{table}
\begin{figure}[htbp]
    \centering
    \includegraphics[width=\linewidth]{images/sim_01_n_SO2_6combies_stat_zoom}
    \caption{ラマンシフト組み合わせ毎の二酸化硫黄測定シミュレーション結果}
    \label{fig:res_6combi}
\end{figure}

\noindent
これより, 統計誤差が最小となるレーザ波長は334.6 nmを使用し, 
酸素アンチストークス散乱波長(318.0 nm)をon波長, 
酸素ストークス散乱波長(353.0 nm)をoff波長とすることがわかった. 

\setcounter{inlineitem}{0}
ここで,先行研究で提案した \inlineitem{itm:st-st} 酸素ストークス散乱波長と窒素ストークス散乱波長の組み合わせと, 
今回新たに提案する \inlineitem{itm:st-ast} 酸素ストークス散乱波長と酸素アンチストークス散乱波長の組み合わせについて, 
各波長におけるラマン散乱波長における二酸化硫黄の吸収断面積を\cref{fig:res_cond_shift_dir}に示す. 
\begin{figure}[htbp]
    \centering
    \includegraphics[width=\linewidth]{images/sim_01_cond_2case}
    \caption{二酸化硫黄の吸収スペクトル上のレーザ波長及びラマン散乱波長}
    \label{fig:res_cond_shift_dir}
\end{figure}

\noindent
\cref{itm:st-st}はoff波長となる酸素ストークス散乱波長(249.3 nm)での吸収断面積が大きい一方で, 
\cref{itm:st-ast}はレーザ波長(318 nm)やoff波長となる酸素ストークス散乱波長(353 nm)の吸収が比較的弱い. 
\cref{itm:st-st}での$\Delta\subrm{\sigma}{SO2}$は$6.83\times10^{-24}\mathrm{m^2}$, 
\cref{itm:st-ast}での$\Delta\subrm{\sigma}{SO2}$は$7.22\times10^{-24}\mathrm{m^2}$と
\cref{itm:st-ast}の方が差分吸収を僅かに大きい. 
また, \cref{itm:st-st}で用いる250 nm前後の波長帯は, \cref{fig:absorptance_spectrum}で示したように, 
硫化水素やオゾンの吸収スペクトルが存在するため, これらの吸収による影響も受けている.  
以上より, 一般にアンチストークス散乱は散乱強度が弱く利用されにくいが, 
本研究で設定した条件下では, 統計誤差の観点から有効であることが示された.

% \section{最適なラマン散乱波長の検討}\label{sec:6combi}
ラマンDIALによる測定で発生する統計誤差は\cref{equ:stat_err}で求められる. 
なお, 背景光による影響は無視している. 
% 式中の$D$はダークカウント, $B_j$は背景光雑音, $F$はディテクターの雑音指数を表す. 
\begin{align}
    \varepsilon_{\mathrm{stat}} 
    &= \dfrac{1}{2\Delta{\sigma}{SO2}\Delta R}\left[
        \sum^2_{i=1}\sum^2_{j=1}
        \frac{1}{P_{ij}}
    \right]
    \label{equ:stat_err}
\end{align}
i=1,2は視線距離$R_1, R_2$であり, 
j=1,2はon波長及びoff波長を表す
\cref{equ:stat_err}より, 
\staterr は$P$と$\Delta\subrm{\sigma}{SO2}$に依存することがわかる.
吸収が過度に強い波長では信号が減衰し$P$が小さくなる一方,
吸収が弱い波長では$\Delta\subrm{\sigma}{SO2}$が小さくなるため,
いずれの場合も統計誤差は増大する.

二酸化硫黄の吸収スペクトルを\cref{fig:absorptance_spectrum}に示す. 
縦軸は消散係数($=\subrm{n}{gas}(R)\times\subrm{\sigma}{gas}(\lambda)$)である. 
二酸化硫黄は紫外領域に吸収スペクトルを持ち, 
300 nm付近の波長では吸収が櫛状に変動する特徴を持つ.
二酸化硫黄測定にラマンDIALを適応する場合, 
\cref{equ:stat_err}の$P$と$\Delta{\sigma}{SO2}$はon波長とoff波長に依存し, 
on波長及びoff波長に相当するラマン散乱波長はレーザ波長に依存する. 
したがって, 信号強度と差分吸収が十分に確保できるレーザ波長が存在するといえる. 
\begin{figure}
    \centering
    \includegraphics[width=\linewidth]{images/sim_01_3gases_absorptance}
    \caption{二酸化硫黄,硫化水素,オゾンの吸収スペクトル}
    \label{fig:absorptance_spectrum}
\end{figure}

酸素分子および窒素分子のラマン散乱波長には,
それぞれストークス散乱波長とアンチストークス散乱波長の2種類が存在し, 
レーザ波長に対して生じる4つのラマン散乱波長から
2波長を選択する場合,
組み合わせは$\,_4\mathrm{C}_2=6$通りとなる.
先行研究ではNd:YAGの第4高調波である266 nmに対する
酸素ストークス散乱波長と窒素ストークス散乱波長による
ラマンDIALについて検討した\cite{Previous_research}. 
しかし, \cref{fig:absorptance_spectrum}に示す通り, 
\ce{SO2}の吸収は240 nm付近や320 nm以降で吸収が弱まるため, 
レーザ波長とラマン散乱波長の関係によっては
アンチストークス散乱波長の利用が優位となる可能性がある. 

以上を踏まえ,
本研究では6通りすべてのラマン散乱波長の組み合わせについて,
二酸化硫黄測定における統計誤差のシミュレーションを行い,
最適なレーザ波長およびラマン散乱波長の組み合わせを検討した.
観測距離は1 kmまで,
気体の濃度分布は三宅島での平均的な火山ガス濃度観測値を参考に, 
二酸化硫黄が高濃度区間で30 ppm, 低濃度区間で0.07 ppm, 
オゾンが観測領域全体で0.005 ppm, 
硫化水素は二酸化硫黄の 1/2 とした.
各気体の吸収断面積の温度・圧力依存性を考慮し, 
ライダー設置高度を1 kmとした.
気体濃度は三宅島における平均的観測値を基に設定した\cite{miyake}.
その他のシステム条件は,
レーザ出力10 mJ,
受信鏡直径0.3 m,
パルス周波数100 Hzで10分間積算,
距離分解能5 mとした.

6通りのラマン散乱波長の組み合わせに対する
二酸化硫黄測定のシミュレーション結果を\cref{fig:res_6combi}に示す.
\begin{figure}[htbp]
    \centering
    \includegraphics[width=\linewidth]{images/sim_01_n_SO2_6combies_stat_zoom}
    \caption{ラマン散乱波長の組み合わせ毎の二酸化硫黄測定シミュレーション結果}
    \label{fig:res_6combi}
\end{figure}
% \begin{table}[htbp]
%     \centering
%     \begin{tabular}{c|c}
%         Combination  & Laser wavelength [nm]\\\hline\hline
%         O2 Stokes, O2 a-Stokes & 334.6 \\
%         N2 Stokes, O2 a-Stokes & 334.6 \\
%         O2 Stokes, N2 a-Stokes & 343.7 \\
%         N2 Stokes, N2 a-Stokes & 338.9 \\
%         N2 a-Stokes, O2 a-Stokes & 343.6 \\
%         N2 Stokes, O2 Stokes & 240.0\\
%     \end{tabular}
%     \caption{Caption}
%     \label{tab:6combi_ls_wl}
% \end{table}

\noindent
その結果,
観測距離700 m地点における統計誤差が最小となる条件は,
レーザ波長334.6 nm,
酸素アンチストークス散乱波長(318.0 nm)をon波長,
酸素ストークス散乱波長(353.0 nm)をoff波長とする波長構成であることがわかった.

\setcounter{inlineitem}{0}
先行研究で提案した
 \inlineitem{itm:st-st} 酸素ストークス散乱波長と窒素ストークス散乱波長の構成と, 
本研究で最適と判断された 
\inlineitem{itm:st-ast} 酸素ストークス散乱波長と酸素アンチストークス散乱波長の構成について, 
各波長における
受信強度の距離特性を\cref{fig:res_cond_power}に示す. 
\begin{figure}[htbp]
    \centering
    \includegraphics[width=\linewidth]{images/sim_01_power.pdf}
    \caption{2構成の各波長における受信強度の距離特性}
    \label{fig:res_cond_power}
\end{figure}

\noindent
\cref{itm:st-st}では,
on波長, off波長ともに吸収が強いため, 高濃度区間での減衰が著しいが, 
\cref{itm:st-ast}では,
不要な吸収を抑え, 遠方での信号強度は十分確保できている. 
また, $\Delta\subrm{\sigma}{SO2}$についても, 
\cref{itm:st-st}は$6.83\times10^{-24} \mathrm{m}^2$であるのに対して, 
\cref{itm:st-ast}は$7.22\times10^{-24} \mathrm{m}^2$と僅かに大きい. 

以上より, 
本研究で設定した条件下では, 
レーザ波長に334.6 nmを使用し, 
酸素アンチストークス散乱波長をon波長,
酸素ストークス散乱波長をoff波長とする
ラマンDIALが統計誤差的に最も優れることが示された. 
これは, 当該波長構成が差分吸収を十分に確保しつつ, 
共存気体による不要な吸収を抑制できるためであり, 
吸収スペクトル形状および共存気体の影響を考慮した上での
アンチストークス散乱波長の利用が有効であることが明らかになった.

\section{干渉気体濃度の推定による干渉誤差の抑制}
ラマンDIALにおいて大気による吸収や測定対象でない様々な気体による吸収は, 
干渉誤差(オフセット誤差)を発生させる. 
これは対象気体濃度を算出するためにDIAL方程式を解く際に,
干渉要素の推定値を含めることで, 
ある程度補正することが可能である. 
\cref{equ:contam_err}は干渉要素を補正するための項であり, 
\cref{equ:n_SO2}から引くことで補正後の対象気体濃度の推定値が得られる.
$\mathrm{gas}=1,2,...,N$は測定時に推定するガス種を表す. 
\begin{align}
    \cntmerr 
    &= \frac{\Delta \alpha_{\mathrm{air}}+\Sigma_{\mathrm{gas}=1}^N n_{\mathrm{gas}}\Delta \sigma_{\mathrm{gas}}}{\Delta \sigma_{\mathrm{SO2}}}\label{equ:contam_err}
\end{align}
硫化水素(\ce{H2S})は\ce{SO2}と並ぶ主要な火山ガス成分であり, 
\ce{SO2}と近い波長帯を吸収するため, 影響を及ぼす. 
特に, 霧島火山のような硫化水素濃度が高い環境では, 
干渉誤差の影響は強くなるため式\cref{equ:contam_err}の補正項を予め考慮する必要がある. 
そこで, \ce{H2S}濃度の推定によってどれだけ干渉誤差の抑制が期待できるかシミュレーションで評価した. 

シミュレーションに用いる火山ガス濃度分布には, 
3章で使用した分布を基に, 
霧島火山における観測値を参考として, 
高濃度区間の \ce{H2S} 濃度を1000 ppmに変更したものを用いた\cite{Kirishima}. 
実際の測定では, 火口の位置の見当はある程度付けることができるため, 
高濃度区間の中心である視線距離500 mに火口があり, 
その前後100 mが高濃度区間だと推定できた状況を想定して補正をかけた. 
補正は一切行わない場合と, 
大気吸収\subrm{\alpha}{aer}
\ce{O3}吸収\subrm{n}{O3}\subrm{\sigma}{O3}, 
\ce{H2S}吸収\subrm{n}{O3}\subrm{\sigma}{O3}について行う場合を比較した. 
そのうち, \ce{O3}の推定濃度分布は, 観測領域全体で0.005 ppmとし, 
\ce{H2S}の推定濃度分布は低濃度区間が0 ppm, 
高濃度区間が0, 500, 600, 800 ppmとしたシミュレーション結果を
\cref{fig:contam_err}に示す. 
\begin{figure}
    \centering
    \includegraphics[width=1\linewidth]{images/sim_02_2_contami_result}
    \caption{\ce{H2S}推定による二酸化硫黄測定への影響}
    \label{fig:contam_err}
\end{figure}

\noindent
まず, 一切補正を行わない場合の\cntmerr は42.4 ppmであるのに対して, 
大気, \ce{O3}についてのみ補正した場合の\cntmerr は41.2 ppmと, 
約1 ppmの誤差改善が表れた. 
そして, さらに\ce{H2S}について高濃度区間で500 ppmと推定した場合の\cntmerr は20.6 ppm, 
600 ppmと推定した場合の\cntmerr は16.5 ppm, 
800 ppmと推定した場合の\cntmerr は8.2 ppmと推定値が実際の設定に近づくほど誤差が改善された.  

霧島火山での\ce{H2S}濃度は低い地点では400 ppm, 高い地点では6000 ppmと報告されており, 
実際の測定で推定することが困難なことを考えると, 
ラマンDIALを\ce{SO2}測定に適用する場合は\ce{H2S}濃度の低い火山環境を選ぶ必要がある\cite{Kirishima}. 
しかし, その他の手法で補助的に\ce{H2S}濃度を測定できる場合や, 
安定した\ce{H2S}濃度環境で事前に妥当な推定値を用意できる場合は, 
干渉誤差を数 ppm程度まで低減して補正できるといえる. 

\section{信号処理方法による統計誤差の改善}
火山の熱水系構造や活動状況, 火口からの距離などにより, 
\ce{SO2}濃度は変動する. 
三宅島における観測値は, 2カ月間(平成13年2, 3月)の平均濃度が2.11 ppmであるのに対して, 
最高値は11.41 ppmが報告されている. 
また, 霧島火山における観測値は, 新燃岳が25.6 ppm, 硫黄山で228 ppmが報告されている. 
提案手法であるラマンDIALにおいて, 
SO2が高濃度, または広範囲に分布する環境で測定すると, 
信号の減衰が大きくなり, 
統計誤差も悪化する.
そのため, \ce{SO2}の濃度変動に伴う統計誤差の悪化を考慮したシステムの設計が必要となる. 

% 前章では, 
% 火口直上の濃度分布を正確にとらえたい場合は
% さらに細かい距離分解能が求められることがわかった. 
% しかし, 統計誤差は\cref{equ:stat_err}より, 
% 距離分解能が大きいほど抑制される相反関係にある. 
DIAL観測では, ライダー信号は細かい距離分解能取得し, 
濃度推定の際は信号強度に合わせて距離分解能の調整が行われる.
距離分解能$\Delta R$は\cref{equ:power}におけるシステムパラメータの中の比例係数であり, 
距離分解能の調整により, 信号加算を行うことで$P$は増大し, 
\cref{equ:stat_err}に示された統計誤差は調整前より減少する. 
本章では, \ce{O2}ストークス散乱と\ce{O2}アンチストークス散乱によるラマンDIALについて, 
距離分解能調整の有無による観測を比較評価した. 

特に火山ガス放出が多い日の観測値を参考に, 
濃度分布は前章で使用したものについて,   
高濃度区間の\ce{SO2}を30 ppm から 100 ppm に変更し, 
距離分解能が5 m と25 mの2つでシミュレーションを行った. 
\cref{fig:res_staterr}にそれぞれのシミュレーション結果を示す. 

\begin{figure}
    \centering
    \includegraphics[width=1\linewidth]{images/sim_02_staterr.pdf}
    \caption{$\Delta R=5, 25$の場合の二酸化硫黄測定シミュレーション結果}
    \label{fig:res_staterr}
\end{figure}

\noindent
\cref{fig:res_staterr}より, どちらの測定でも, 
視線距離が300 m 及び700 m周辺にある濃度の切り替わりで誤差が発生するが, 
$\Delta R$が25 mの測定は追従性の不足がより顕著である. 
一方で, 統計誤差は$\Delta R$が5 mの場合, 697.5m地点で13.4 ppmであるのに対して, 
$\Delta R$が25 mの場合, 697.5m地点で1.16 ppmに改善された.
これは, 設定値の100 ppmに対して10 \%以下を十分に満たしているため, 
距離分解能の調整で実測定に耐えるシステムとなることが示された. 
そのため, 実観測では高距離分解能でライダー信号を取得した上で, 
濃度推定時に信号強度や濃度勾配に応じて
距離分解能を動的に変更する処理が有効であると考えられる.

\section{プルームモデルの導入}
前章までは, ある区間で火山ガス濃度が一定の理想モデルで検討を行った. 
本章では, より現実的な火山ガス拡散を考慮するため, 
プルームモデルを導入し, ラマンDIALによる\ce{SO2}濃度分布観測の適用可能性を評価した. 
プルームモデルは, 流体の移流拡散方程式の解析解であり, 
排煙による大気汚染の予測によく用いられる. 
% 煙源からの風下距離$x$, 風向に直交する水平横断距離$y$ における地表のガス濃度$n_{\mathrm{gas}}$は\cref{equ:plume}で表される.
% \begin{align}
%     n_{\mathrm{gas}}(x, y) = \dfrac{Q}{2\pi\omega_{y}(x)\omega_{z}(x)u}
%     \exp{\left\{
%         -\frac{y}{2\omega_y^2(x)}
%     \right\}}
%     \exp{\left\{
%         -\frac{H_\mathrm{e}^2}{2\omega_z^2(x)}
%     \right\}}\label{equ:plume}
% \end{align}
% 式中の$Q$は煙の発生源の排出強度, 
% $u$は風速, 
% $\omega_{y}, \omega_{z}$はそれぞれ水平および鉛直方向の煙流拡散幅, 
% $H_\mathrm{e}$は有効煙突高さである. 
プルームモデルはプルームの拡散幅に依存する関数であり, 
拡散幅は煙源からの風下距離と大気安定度に従って, 
煙源から遠く, 大気が不安定であるほど拡散幅は大きくなる. 
今回, 大気安定度の分類とプルームの拡散幅は, 
Pasquill-Gifford型の大気拡散モデルに基づいて設定した. 

前章で, 距離分解能調整の有効性であることが示されたため, 
\ce{SO2}煙流の分布をより細かく得られるように距離分解能をより小さくしたい. 
そのため, ライダーパラメータは前章までのものを使用しながら, 距離分解能のみ
レーザのパルス時間幅で10 nsec相当の1.5mに向上させてシミュレーションした. 

\cref{fig:situation_plume}に, シミュレーションに用いる火山ガス濃度の水平面分布を示す. 
気象条件の例として, 風速 2 m/sec かつ快晴とする拡散しやすい条件と, 
風速 10 m/sec かつ曇天とする拡散しにくい条件を与え,  
風向はライダー視線に対して垂直に横切る方向とした.
また, 火口配置については, ライダー視線が火口直上を横切る場合と, 
火口から風下 25 m 付近を横切る場合, 
火口から風下 50 m 付近を横切る場合の3通りを想定した. 

% \cref{fig:situation_plume}に高度1000mの地表面における
% (a)火山ガスの平面分布と, 
% (b)ライダー視線上の二酸化硫黄, 硫化水素, オゾンの濃度分布
% を示す. \cref{fig:plum_map}においてライダー視線はX軸上を通る. 
% プルームモデルに与える煙流の拡散幅はパスキルの大気安定分類と, 
% ギフォードの水平・鉛直拡散幅推定によるpasquill-gifford線図を距離方向に拡張した 
% \cref{fig:pg_spread}を用いた\cite{pasquill}\cite{gifford}. 

% \begin{figure}[htbp]
%   \centering
%   \begin{subfigure}{0.49\linewidth}
%     \centering
%     \includegraphics[width=\linewidth]{images/sim_04_field_spread_leteral}
%     \caption{水平拡散幅}
%     \label{fig:pg_spread_leteral}
%   \end{subfigure}
%   \hfill
%   \begin{subfigure}{0.49\linewidth}
%     \centering
%     \includegraphics[width=\linewidth]{images/sim_04_field_spread_vertical}
%     \caption{鉛直拡散幅}
%     \label{fig:pg_spread_vertical}
%   \end{subfigure}
%   \caption{pasquill-gifford線図}
%   \label{fig:pg_spread}
% \end{figure}

\begin{figure}
    \centering
    \begin{tabular}{@{}c c@{}}
        %------------------------
        % 図1(左上)
        %------------------------
        \begin{minipage}[t]{0.5\linewidth}
            \centering
            \includegraphics[width=\linewidth]{images/sim_04_env_image_1}
            \subcaption{弱拡散:水平面分布}
            \label{fig:plume_low_horizontal}
        \end{minipage}
        &
        %------------------------
        % 図2(右上)
        %------------------------
        \begin{minipage}[t]{0.5\linewidth}
            \centering
            \includegraphics[width=\linewidth]{images/sim_04_env_LoS_2}
            \subcaption{弱拡散:視線上分布}
            \label{fig:plume_low_los}
        \end{minipage}
        \\

        %------------------------
        % 図3(左下)
        %------------------------
        \begin{minipage}[t]{0.5\linewidth}
            \centering
            \includegraphics[width=\linewidth]{images/sim_04_env_image_2}
            \subcaption{強拡散:水平面分布}
            \label{fig:plume_high_horizontal}
        \end{minipage}
        &
        %------------------------
        % 図4(右下)
        %------------------------
        \begin{minipage}[t]{0.5\linewidth}
            \centering
            \includegraphics[width=\linewidth]{images/sim_04_env_LoS_1}
            \subcaption{強拡散:視線上分布}
            \label{fig:plume_high_los}
        \end{minipage}
    \end{tabular}

    %========================
    % 図全体キャプション
    %========================
    \caption{プルームモデルに従う火山ガス濃度分布の設定}
    \label{fig:situation_plume}

\end{figure}

弱拡散場でのラマンDIAL測定は, 
\cref{fig:res_plume_meas_widespread}に示すように,
手前側の2つのプルームは距離分解能不足によるピークの取りこぼしが発生しているが, 
奥側のプルームは, 十分に濃度分布の変化を捉えられている.
それに対して, \cref{fig:res_plume_meas_narrowspread}に示す
強拡散場でのラマンDIAL測定は,
手前側の1つ目のプルームはピークの取りこぼしが発生したものの, 
2つ目以降のプルームについては, 濃度分布の変化を捉えられている.
\begin{figure}[htbp]
    \centering
    \includegraphics[width=\linewidth]{images/sim_04_plume_meas_2}
    \caption{弱拡散場の測定シミュレーション結果}
    \label{fig:res_plume_meas_widespread}
\end{figure}
\begin{figure}[htbp]
    \centering
    \includegraphics[width=\linewidth]{images/sim_04_plume_meas_1}
    \caption{強拡散場の測定シミュレーション結果}
    \label{fig:res_plume_meas_narrowspread}
\end{figure}

距離分解能を1.5 mに向上させたが弱拡散で火口配置が近い場合や, 
強拡散で火口直上といった場合など, 
プルームの濃度分布の変化に測定が追従しきれない場合が明らかになった. 
しかし, 多くのケースでプルームの濃度分布の変化を捉えられており,
距離分解能が十分な測定点では5章で示した信号加算を用いることで
火山ガス濃度分布観測が期待できるといえる. 

\section{まとめ}
本研究では, 火山ガスに含まれる\ce{SO2}の濃度分布を
遠隔かつ連続的に測定する手法としてラマンDIALに着目し,
火山観測への適用可能性についてシミュレーションによる検討を行った. 

まず, 酸素分子と窒素分子によるラマン散乱波長が
取りうる6通りの組み合わせについて, 
統計誤差が最小となる組み合わせ及びレーザ波長を評価した.  
結果として, 
設定した条件下において, レーザ波長は334.6 nmを使用し,
酸素アンチストークス散乱波長(318.0 nm)をon波長, 
酸素ストークス散乱波長(353.0 nm)をoff波長とするラマンDIALが
最適であることが明らかになった. 
一般的に散乱強度が弱く利用されにくいアンチストークス散乱が, 
条件次第では有効な選択肢となることが示された. 

次に, \ce{SO2}測定に影響を与える\ce{H2S}の濃度を測定時に推定することで
得られる干渉誤差の補正効果について評価した. 
\ce{H2S}濃度の設定値が1000oppmに対して, 
800 ppmと推定することで干渉誤差は8.2 ppmまで低減されることが示された. 
実際の測定では, \ce{H2S}濃度が低い場合や, 他の手法で補助的に\ce{H2S}濃度を測定できる場合といった条件付で
\ce{H2S}の干渉を補正可能である. 

最後に, より現実的な火山ガス拡散としてプルームモデルをシミュレーションに導入し, 
ラマンDIALによる\ce{SO2}濃度分布測定の適用可能性を評価した. 
火口からの風下距離が近いほど局所的に高濃度となる濃度分布となるため, 
距離分解能が1.5mのラマンDIALで測定する場合, 
強拡散場や火口から離れた地点の\ce{SO2}分布の変化を捉えきれることがわかった. 
今回検証した統計誤差改善のための距離分解能の調整を, 
濃度勾配に応じて動的に変更する処理系が有効だと考えられる. 

% Print bibliography from reference.bib using biblatex with custom heading
% \printbibliography[title=参考文献]
\begin{thebibliography}{9}\footnotesize

  
  % \bibitem{PowerEqu}  T.J.McGee, AGU Monograph, 1993, p.956.
  
  \bibitem{DIALEqu} S.Ismail, Appl.Opt., 28(17)(1989), 3604.
  \bibitem{PowerEqu} McGee, Thomas J., AGU Meeting,1993,956
  % title     = {Raman DIAL measurements of stratospheric ozone in the presence of volcanic aerosols},
  
  \bibitem{Previous_research} 伊藤, 東京都立大学システムデザイン学部特別研究, 2022.

  \bibitem{miyake} 内閣府, 三宅島火山ガスに関する検討会報告書, 2003.

  \bibitem{Kirishima} 大場武他, 日本火山, 42(1)(1997), 1-15.

  % \bibitem{ramanXS} 清水浩他, 応用物理, 42(1973), 889-898.

% \bibitem{pasquill}
% F.Pasquill, \textit{Atmospheric Diffusion}, D.Van~Nostrand, 1962.

% \bibitem{gifford}
% F.A.Gifford Jr., Nuclear Safety, 2(4)(1961), 47-51.

\end{thebibliography}


\end{document}
