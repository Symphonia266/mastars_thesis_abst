\section{まとめ}
本研究では, 火山ガスに含まれる\ce{SO2}の濃度分布を
遠隔かつ連続的に測定する手法としてラマンDIALに着目し,
火山観測への適用可能性についてシミュレーションによる検討を行った. 

まず, 酸素分子と窒素分子によるラマン散乱波長が
取りうる6通りの組み合わせについて, 
統計誤差が最小となる組み合わせ及びレーザ波長を評価した.  
結果として, 
設定した条件下において, レーザ波長は334.6 nmを使用し,
酸素アンチストークス散乱波長(318.0 nm)をon波長, 
酸素ストークス散乱波長(353.0 nm)をoff波長とするラマンDIALが
最適であることが明らかになった. 
一般的に散乱強度が弱く利用されにくいアンチストークス散乱が, 
条件次第では有効な選択肢となることが示された. 

次に, \ce{SO2}測定に影響を与える\ce{H2S}の濃度を測定時に推定することで
得られる干渉誤差の補正効果について評価した. 
\ce{H2S}濃度の設定値が1000oppmに対して, 
800 ppmと推定することで干渉誤差は8.2 ppmまで低減されることが示された. 
実際の測定では, \ce{H2S}濃度が低い場合や, 他の手法で補助的に\ce{H2S}濃度を測定できる場合といった条件付で
\ce{H2S}の干渉を補正可能である. 

最後に, より現実的な火山ガス拡散としてプルームモデルをシミュレーションに導入し, 
ラマンDIALによる\ce{SO2}濃度分布測定の適用可能性を評価した. 
火口からの風下距離が近いほど局所的に高濃度となる濃度分布となるため, 
距離分解能が1.5mのラマンDIALで測定する場合, 
強拡散場や火口から離れた地点の\ce{SO2}分布の変化を捉えきれることがわかった. 
今回検証した統計誤差改善のための距離分解能の調整を, 
濃度勾配に応じて動的に変更する処理系が有効だと考えられる. 
