\section{はじめに}
日本は世界全体の約7\%を占める活火山が存在する火山大国であり, 
火山ガスによる人的被害を防ぐために監視が行われている. 
その中でも, 二酸化硫黄は火山ガスの中でも特に強い毒性を持ち, 
2 ppmが許容濃度として設定されており, 
三宅火山をはじめとする様々な火山で測定が行われている. 

火山ガス観測には, 
ガスと触媒との電気化学反応量から濃度を測定する電気化学式センサが広く用いられている. 
しかし, この手法は定点観測を前提としているため, 
気体濃度の空間分布を把握することが困難である. 
さらに, 測定器が定常状態に達するまで濃度を評価できないという制約があり, 
時間的に連続した観測が困難であるという課題を有している.
% 火山ガスの有害性から観測者の安全を確保する遠隔観測手法が求められている. 
そこで,  微量気体の遠隔かつ分布観測が可能な手法として, 
差分吸収ライダー(DIAL:Differential Absorption Lidar)が期待されている. 

DIALは測定気体の吸収が異なる2波長レーザを照射し, 
その受信量の差から濃度分布を推定する手法である. 
電気化学式センサと比較して, 遠隔で時間的に連続して, 周辺の濃度分布を測定可能という点で優れている. 
本研究で提案するラマンDIALは, 
% システムに用いる2波長を送信光によるラマン散乱光とし, 
送信レーザの波長数を削減することでシステムの小型化が可能である. 
% 本研究では, 火山ガス中の二酸化硫黄濃度をラマンDIALで観測することを目標とした.